\documentclass[12pt, letterpaper]{article}
\usepackage[utf8]{inputenc}
\usepackage{graphicx}
\usepackage{mathpartir}
\usepackage{textcomp}
\usepackage{gensymb}
\usepackage[libertine]{newtxmath}
\usepackage[margin=1.0in]{geometry}
\usepackage[tt=false, type1=true]{libertine}

\begin{document}

\title{
  \textsc{teler}: A Streamlined Version Control System\\
  \large Project Report
}

\author{
  Hadley Luker \and Zachary J. Susag
}

\maketitle

\section{Project Overview}
\label{sec:overview}

During their education, most computer science students must learn to use a
version control system (VCS). Although these systems see wide use in research
and industry for software development and production, in educational contexts
they typically service a narrower use case: saving, submitting, and tracking
changes in programming assignments. This limited use case causes many students
to become overwhelmed when confronted with the numerous and powerful commands
available to modern VCSes such as Git and Mercurial. For example, the typical
workflow of saving changes to a remote repository in Git proceeds as follows:

\begin{enumerate}
  \item \texttt{git add .}
  \item \texttt{git commit -m "Fixed bug"}
  \item \texttt{git push}
\end{enumerate}

These three commands are practically an idiom that students are habitually
trained into, and deviations from this pattern are rare and incite confusion.
For example, if the \texttt{-m} is omitted from the end of \texttt{git commit},
the terminal will run the Vim text editor by default in order for the user to
write a commit message. Due to Vim's unintuitive keybindings and minimal user
interface, students confronted with this sudden change in environment may make
errors in the editor or find themselves unable to exit and subsequently close
the entire hosting terminal window.

For these reasons, we decided to develop \textsc{teler}: a streamlined version
control system for the student use case. Our system exposes exactly five
commands: repository initialization, pushing changes with a required commit-like
summary to a remote, pulling changes from a remote, reverting changes to a
specific version, and viewing version history. This interface condenses the
common add-commit-push / pull command sequence at the expense of requiring all
files to be pushed or pulled at once. However, this has positive effects:
it encourages users to push frequently and document important changes as they
are made.

***Evaluation summary.***

\section{Design \& Implementation}
\label{sec:desimp}
The design of our program is centered around two main commands and three
components: the command line interface, the shadow directory which contains all
changes recorded by \textsc{teler}, and the auxiliary data structures and
compression that enable \textsc{teler} to be more time and space efficient.

\subsection{Design}
\label{subsec:design}
The use of textsc{teler} depends mostly on two key commands, \texttt{push} and
\texttt{pull}.

\subsubsection{texttt{push}}
\label{subsubsec:push}
The act of pushing in a \textsc{teler} repository saves new changes in a
space-efficient manner within the \textsc{teler} shadow directory, described
later. The program first loads the previous commit into memory and compares the
hashes of the files it finds to those in the current working directory at those
positions. If they differ, \textsc{teler} discards the changes *** i forgot at
*** this point ***

\subsubsection{texttt{pull}}
\label{subsubsec:pull}

\subsection{Shadow Directory}
\label{subsec:shadowdir}

\subsection{Command-Line Interface}
\label{subsec:cli}

The command-line interface is a simple argument parser made using GNU C's
\texttt{argp} interface. This interface automatically generates help and syntax
information for display on the command line through the assignment of certain
global variables and "option vectors", a structure which organizes and
configures command line arguments. However, our implementation is slightly
modified from the GNU standard of C argument morphology. The GNU practice is to
precede all arguments on the command line with a number of hyphens: one for
single-character arguments (e.g. \texttt{-v}) and two for long arguments (e.g.
\texttt{--verbose}). We chose to let the first argument be hyphen-less. (For
differentation's sake, we refer to this "first argument" as the "command".) This
reflects more closely the practices used by Git for their command line
interface, which we suspect will be more familiar to students. This decision
caused us to parse the first argument through an if--else if chain and instead
use \texttt{argp}'s parser function to parse arguments given to the command
itself.

\subsection{Data Structures}
\label{subsec:datastructures}

\subsubsection{Hash Table}
\label{subsubsec:datastructures}

The hash table indexes all files within the repository, in addition to certain
directory information and commits. Any data structure could feasibly be used for
this, but since its primary purpose is for lookup, a hash table was chosen.

\subsubsection{Tree}
\label{subsubsec:tree}

The tree organizes hash table entries by their file system structure. It enables
\texttt{teler} to save or reconstruct the version of the repository that was
saved in the latest commit.

\subsection{Compression}
\label{subsec:compression}

As \textsc{teler} saves every version of a file that was added in a change, it
becomes important to utilize every means of space-reduction we have available to
us. Therefore, all files and commits within a \textsc{teler} repository are
compressed. We utilize the \texttt{zlib} library to accomplish this. *** How it
do and how we use ***



\section{Evaluation}
\label{sec:evaluation}
To evaluate \textsc{teler} we compared its speed in generating commits
through \textsc{teler push} with Git. For hardware we used an Intel
Core i5-3320M at 3.3GHz with 8GB of RAM with a 320GB spinning hard
disk. The workstation we used to evaluate \textsc{teler} was running
Arch Linux with kernal version x86_64 Linux 4.16.8-1-ARCH. We
collected the running times of Git and \textsc{teler} using the
\texttt{time} command.

To fairly assess each program we generated a random directory with a
maximum depth per directory of 3, up to 4 first-level directories, and
up to 6 maximum children per directory. We then recorded the time it
took for each program to make a commit of the entire directory. This
was done ten times with a new subdirectory of the same specifications
being made, increasing the size of the total directory each time.
During this process we also recorded the total size of the directory
in bytes.


\end{document}
