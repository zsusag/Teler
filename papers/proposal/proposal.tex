\documentclass[12pt, letterpaper]{article}
\usepackage[utf8]{inputenc}
\usepackage{graphicx}
\usepackage{mathpartir}
\usepackage{textcomp}
\usepackage{gensymb}
\usepackage[libertine]{newtxmath}
\usepackage[margin=1.0in]{geometry}
\usepackage[tt=false, type1=true]{libertine}

\begin{document}

\title{
  \textsc{teler}: A Streamlined Version Control System\\
  \large Project Proposal
}

\author{
  Hadley Luker \and Zachary J. Susag
}

\maketitle

\section{Problem Outline}
\label{sec:problem}
During their computer science education, many students utilize version
control systems for class assignments, professional work, and personal
projects. However, students rarely use the entire set of functionality
available from version control systems, either due to its
unintuitiveness or its inapplicability to their common use cases. For
example, a common workflow for a student using Git entails the
following:
\begin{enumerate}
\item \texttt{git add .}
\item \texttt{git commit -m "bad commit message"}
\item \texttt{git push}
\end{enumerate}
As you can see, the process to push all changes made in a local
repository to a remote takes three separate commands: the addition of changes to a
staging area, marking these changes with a commit message, and then
pushing the changes to the remote. Many students are not familiar with
commands outside of this range, and in fact even versions of these
commands with different flags or arguments are unfamiliar. For
instance, if a student runs \texttt{git commit} without any flags, Git
automatically opens Vim by default. Vim's unintuitive keybindings and
minimal user interface often cause unfamiliar students to panic and
quit the terminal entirely. Other common problems include manually
resolving merge conflicts, failing to add all the desired files to the
staging area, and difficulty reverting a repository to a specific
previous commit.

One conceptual barrier a student must overcome is the staging area.
Given the above workflow, the staging area seems
superfluous as the main goal of a VCS is to synchronize
changed files across systems and to label said changes.

Overall, git dilutes its functionality among numerous, unintuitively
named commands. As such, version control systems like git have a steep
learning curve before a user could fully benefit from the power they provide.


\section{Approach}
\label{sec:approach}

We propose a streamlined, minimal, distributed version control system,
the main emphasis of which is on the core of a typical student's
workflow: synchronization with a remote repository and
change tracking. Instead of spreading functionality out among several
commands, like git, our proposed VCS, dubbed \textsc{teler} will have
two main commands for pushing and pulling from a remote as well as
auxiliary commands for reverting to previous commits, branching, etc.

All of our proposed commands will operate off of reasonable defaults
based upon the student's common use case. For pushing to a remote we
assume that the user wants to send all changed files to the first
added remote, named \textit{origin} in git. This condenses the
``adding'', ``committing'' and ``pushing'' steps in git to a single
command in \textsc{teler}. It should be noted that this approach
eliminates the staging area entirely. In order for a user to send
specific files instead of all changed files they would need to specify
via a flag. For pulling, we make the assumption that the user would
like to receive all files from the remote directory, just like git.

By collapsing the ``add'', ``commit'', ``push'' routine into a single
step, symmetry is created between the actions of pushing and pulling.
We argue this symmetry provides a more natural, intuitive workflow.

To measure how successful \textsc{teler} is we would like to ideally
have a student test the system out and see how natural it is. If it
comes intuitively while still providing the essential power that git
does, we would consider \textsc{teler} a success.

\section{Problem Solving Strategies}
\label{sec:strategies}

The two problem solving strategies that \textsc{teler} will use are
networking and persistent storage. 

\subsection{Networking}
\label{sec:networking}
\textsc{teler} will be a distributed VCS, meaning that a copy of the
full history of the repository will be stored locally. This makes it
possible to use \textsc{teler} without the need of a central server,
instead relying on peer-to-peer communications. Upon a push/pull, files
will be sent to or received from the network location of the remote.
We hope to be able to make this connection by opening a \texttt{ssh}
socket to the remote repository and transmitting a serialized version
of the commit being sent or received. It should also be noted that
this serialized version would be encrypted via \texttt{ssh}.
Essentially, we hope to achieve the same distributed functionality as
git.

\subsection{Persistent Storage}
\label{sec:storage}

A crucial part of \textsc{teler} will be detecting whether a file has
been changed since the latest commit. As such, when the user is ready
to push changes to a remote \textsc{teler} will have to recursively
scan the directory for new, changed, or removed files from the
previously stored commit. Once these files are identified they will
need to be added to the shadow directory (\texttt{.git} in git) in a
manner in which it is easy and fast to find the commit. To achieve
this we will store the file under its SHA-1 hash, along with
other commit metadata, and subsequently store the compressed file in
an objects directory. To minimize space we will compress all files
using \texttt{zlib}. It will also potentially be necessary to store
metadata information of these files such as owners, permissions,
and modification times. In essence, we will be creating a incredibly
minimal filesystem to efficiently store the history of the repository.

\section{Implementation Outline}
\label{sec:implementation}
We identify three major components of \textsc{teler}: the shadow
directory structure, the network transmission protocol, and
determining differences between file versions.

For the shadow directory we plan to use a similar structure as git.
In git, within the \texttt{.git} directory there is a file for
each branch which lists the latest commit, configuration information
such as remote locations, and all of the objects which represent the
full history of the repository. An object is either a tree or a blob,
both compressed using \texttt{zlib}. A tree represents a directory and
can contain trees and blobs. A blob is simply a compressed version of
a file present in the directory. The objects will be stored according
to the SHA-1 hash of the file to allow for both rapid checking of
differences as well as efficient search within the object directory.
Note that this structure naturally lends itself to a recursive
approach. When a user wishes to revert their directory to a previous
commit all that is necessary is to find the commit number and decompress
the commit log which will list the head of the tree. From there, all
that is necessary is to descend the tree structure and write the
decompressed blobs to the working directory. As a result of this
organization no file will be stored multiple times within the
directory; however, multiple references to a blob may exist. In essence, the shadow
directory is simply a directed, acyclic graph in which the nodes are
the commits and the edges preserve precedence among said commits.

For the networking portion we will use a similar peer-to-peer
networking system as in the chat room lab. On each system
\textsc{teler} will run in the background continuously accepting
connections from other instances of \textsc{teler} on other systems.
When a push command is executed a socket to the default remote
location, as specified by the user, will be opened. Once a connection
has been established the local instance of \textsc{teler} will send
first the metadata information about the latest commit, then the
entire tree of blobs concerning that commit. The remote instance of
\textsc{teler} will receive these changes and store them within its
shadow directory using the same organization mechanism. Once the
changes are received the connection is severed and the remote
directory will be changed to reflect the differences. We hope to be
able to use the \texttt{ssh} protocol to send these changes in an
efficient manner.

Lastly, in order to determine whether a change to a file has been
made, first \textsc{teler} must
traverse the tree of blobs from the latest commit, generate the
hashes of the files in the working directory, and compare them to
those of the latest commit. If the hashes are different then the file
has been changed and this change should be reflected in the shadow
directory when the push command is executed. Along with this component
are commit reversion utilities and creating diffs of changed files.

By the end of week one we hope to have fleshed out the shadow
directory structure and being able to add commits to the shadow
directory. By the end of week two we hope to be able to have
\textsc{teler} efficiently find changed files, create commits
automatically, and update the shadow directory as appropriate. The
final week we will be adding networking functionality to allow for
synchronization across machines.

We expect that the greatest difficulty will be creating this shadow
directory. With the amount of file operations along with metadata
tracking it is probable that some data will be lost among the first
few versions. It also may be necessary to simplify the structure and
attempt to enhance it later. Furthermore, the serialization of the
commits into a format that is easily sent and received could fail,
especially if we combine that with using \texttt{ssh}. Figuring out
how to have the remote repository determine where it should place the
received commit could also prove difficult.
\end{document}
